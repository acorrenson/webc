\documentclass[french, twocolumn]{article}
\usepackage{babel}[french]
\usepackage[T1]{fontenc}
\usepackage[margin=1in]{geometry}
\begin{document}
    \title{Webc : a web framework for the C programming language}
    \author{Jdrprod \and Anima Libera \and 4skl}
    \maketitle
    \begin{abstract}
        Le 21ième siècle a vu l'essor des techonologies web et les frameworks spécialisés dans la conception de sites internets se multiplient a une vitesse déraisonnable. \textbf{Webc} propose un retour aux saintes sources de la programmation en offrant un outil simple et efficace pour écrire des sites webs statics en C.
    \end{abstract}
    \section{Styles de programmation C}
    \begin{itemize}
        \item Organisation des sources
        \item Modèle Orienté Objet
        \item Gestion de la compilation
    \end{itemize}
    \section{Des bits à travers le réseau}
    \begin{itemize}
        \item Sockets BSD
        \item Algos et structures pour la lecture et l'écriture en buffers
    \end{itemize}
    \section{HTTP}
    \begin{itemize}
        \item Implémentation (partielle) du protocol HTTP
        \item Routage ?
    \end{itemize}
    \section{\textit{Templating}}
    \begin{itemize}
        \item Moteur de template
        \item Parseur HTML
        \item Représentation/parcours des sources
    \end{itemize}
    \section{Conclusion et travaux futures}
\end{document}